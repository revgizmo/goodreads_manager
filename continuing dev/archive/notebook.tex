
% Default to the notebook output style

    


% Inherit from the specified cell style.




    
\documentclass[11pt]{article}

    
    
    \usepackage[T1]{fontenc}
    % Nicer default font (+ math font) than Computer Modern for most use cases
    \usepackage{mathpazo}

    % Basic figure setup, for now with no caption control since it's done
    % automatically by Pandoc (which extracts ![](path) syntax from Markdown).
    \usepackage{graphicx}
    % We will generate all images so they have a width \maxwidth. This means
    % that they will get their normal width if they fit onto the page, but
    % are scaled down if they would overflow the margins.
    \makeatletter
    \def\maxwidth{\ifdim\Gin@nat@width>\linewidth\linewidth
    \else\Gin@nat@width\fi}
    \makeatother
    \let\Oldincludegraphics\includegraphics
    % Set max figure width to be 80% of text width, for now hardcoded.
    \renewcommand{\includegraphics}[1]{\Oldincludegraphics[width=.8\maxwidth]{#1}}
    % Ensure that by default, figures have no caption (until we provide a
    % proper Figure object with a Caption API and a way to capture that
    % in the conversion process - todo).
    \usepackage{caption}
    \DeclareCaptionLabelFormat{nolabel}{}
    \captionsetup{labelformat=nolabel}

    \usepackage{adjustbox} % Used to constrain images to a maximum size 
    \usepackage{xcolor} % Allow colors to be defined
    \usepackage{enumerate} % Needed for markdown enumerations to work
    \usepackage{geometry} % Used to adjust the document margins
    \usepackage{amsmath} % Equations
    \usepackage{amssymb} % Equations
    \usepackage{textcomp} % defines textquotesingle
    % Hack from http://tex.stackexchange.com/a/47451/13684:
    \AtBeginDocument{%
        \def\PYZsq{\textquotesingle}% Upright quotes in Pygmentized code
    }
    \usepackage{upquote} % Upright quotes for verbatim code
    \usepackage{eurosym} % defines \euro
    \usepackage[mathletters]{ucs} % Extended unicode (utf-8) support
    \usepackage[utf8x]{inputenc} % Allow utf-8 characters in the tex document
    \usepackage{fancyvrb} % verbatim replacement that allows latex
    \usepackage{grffile} % extends the file name processing of package graphics 
                         % to support a larger range 
    % The hyperref package gives us a pdf with properly built
    % internal navigation ('pdf bookmarks' for the table of contents,
    % internal cross-reference links, web links for URLs, etc.)
    \usepackage{hyperref}
    \usepackage{longtable} % longtable support required by pandoc >1.10
    \usepackage{booktabs}  % table support for pandoc > 1.12.2
    \usepackage[inline]{enumitem} % IRkernel/repr support (it uses the enumerate* environment)
    \usepackage[normalem]{ulem} % ulem is needed to support strikethroughs (\sout)
                                % normalem makes italics be italics, not underlines
    

    
    
    % Colors for the hyperref package
    \definecolor{urlcolor}{rgb}{0,.145,.698}
    \definecolor{linkcolor}{rgb}{.71,0.21,0.01}
    \definecolor{citecolor}{rgb}{.12,.54,.11}

    % ANSI colors
    \definecolor{ansi-black}{HTML}{3E424D}
    \definecolor{ansi-black-intense}{HTML}{282C36}
    \definecolor{ansi-red}{HTML}{E75C58}
    \definecolor{ansi-red-intense}{HTML}{B22B31}
    \definecolor{ansi-green}{HTML}{00A250}
    \definecolor{ansi-green-intense}{HTML}{007427}
    \definecolor{ansi-yellow}{HTML}{DDB62B}
    \definecolor{ansi-yellow-intense}{HTML}{B27D12}
    \definecolor{ansi-blue}{HTML}{208FFB}
    \definecolor{ansi-blue-intense}{HTML}{0065CA}
    \definecolor{ansi-magenta}{HTML}{D160C4}
    \definecolor{ansi-magenta-intense}{HTML}{A03196}
    \definecolor{ansi-cyan}{HTML}{60C6C8}
    \definecolor{ansi-cyan-intense}{HTML}{258F8F}
    \definecolor{ansi-white}{HTML}{C5C1B4}
    \definecolor{ansi-white-intense}{HTML}{A1A6B2}

    % commands and environments needed by pandoc snippets
    % extracted from the output of `pandoc -s`
    \providecommand{\tightlist}{%
      \setlength{\itemsep}{0pt}\setlength{\parskip}{0pt}}
    \DefineVerbatimEnvironment{Highlighting}{Verbatim}{commandchars=\\\{\}}
    % Add ',fontsize=\small' for more characters per line
    \newenvironment{Shaded}{}{}
    \newcommand{\KeywordTok}[1]{\textcolor[rgb]{0.00,0.44,0.13}{\textbf{{#1}}}}
    \newcommand{\DataTypeTok}[1]{\textcolor[rgb]{0.56,0.13,0.00}{{#1}}}
    \newcommand{\DecValTok}[1]{\textcolor[rgb]{0.25,0.63,0.44}{{#1}}}
    \newcommand{\BaseNTok}[1]{\textcolor[rgb]{0.25,0.63,0.44}{{#1}}}
    \newcommand{\FloatTok}[1]{\textcolor[rgb]{0.25,0.63,0.44}{{#1}}}
    \newcommand{\CharTok}[1]{\textcolor[rgb]{0.25,0.44,0.63}{{#1}}}
    \newcommand{\StringTok}[1]{\textcolor[rgb]{0.25,0.44,0.63}{{#1}}}
    \newcommand{\CommentTok}[1]{\textcolor[rgb]{0.38,0.63,0.69}{\textit{{#1}}}}
    \newcommand{\OtherTok}[1]{\textcolor[rgb]{0.00,0.44,0.13}{{#1}}}
    \newcommand{\AlertTok}[1]{\textcolor[rgb]{1.00,0.00,0.00}{\textbf{{#1}}}}
    \newcommand{\FunctionTok}[1]{\textcolor[rgb]{0.02,0.16,0.49}{{#1}}}
    \newcommand{\RegionMarkerTok}[1]{{#1}}
    \newcommand{\ErrorTok}[1]{\textcolor[rgb]{1.00,0.00,0.00}{\textbf{{#1}}}}
    \newcommand{\NormalTok}[1]{{#1}}
    
    % Additional commands for more recent versions of Pandoc
    \newcommand{\ConstantTok}[1]{\textcolor[rgb]{0.53,0.00,0.00}{{#1}}}
    \newcommand{\SpecialCharTok}[1]{\textcolor[rgb]{0.25,0.44,0.63}{{#1}}}
    \newcommand{\VerbatimStringTok}[1]{\textcolor[rgb]{0.25,0.44,0.63}{{#1}}}
    \newcommand{\SpecialStringTok}[1]{\textcolor[rgb]{0.73,0.40,0.53}{{#1}}}
    \newcommand{\ImportTok}[1]{{#1}}
    \newcommand{\DocumentationTok}[1]{\textcolor[rgb]{0.73,0.13,0.13}{\textit{{#1}}}}
    \newcommand{\AnnotationTok}[1]{\textcolor[rgb]{0.38,0.63,0.69}{\textbf{\textit{{#1}}}}}
    \newcommand{\CommentVarTok}[1]{\textcolor[rgb]{0.38,0.63,0.69}{\textbf{\textit{{#1}}}}}
    \newcommand{\VariableTok}[1]{\textcolor[rgb]{0.10,0.09,0.49}{{#1}}}
    \newcommand{\ControlFlowTok}[1]{\textcolor[rgb]{0.00,0.44,0.13}{\textbf{{#1}}}}
    \newcommand{\OperatorTok}[1]{\textcolor[rgb]{0.40,0.40,0.40}{{#1}}}
    \newcommand{\BuiltInTok}[1]{{#1}}
    \newcommand{\ExtensionTok}[1]{{#1}}
    \newcommand{\PreprocessorTok}[1]{\textcolor[rgb]{0.74,0.48,0.00}{{#1}}}
    \newcommand{\AttributeTok}[1]{\textcolor[rgb]{0.49,0.56,0.16}{{#1}}}
    \newcommand{\InformationTok}[1]{\textcolor[rgb]{0.38,0.63,0.69}{\textbf{\textit{{#1}}}}}
    \newcommand{\WarningTok}[1]{\textcolor[rgb]{0.38,0.63,0.69}{\textbf{\textit{{#1}}}}}
    
    
    % Define a nice break command that doesn't care if a line doesn't already
    % exist.
    \def\br{\hspace*{\fill} \\* }
    % Math Jax compatability definitions
    \def\gt{>}
    \def\lt{<}
    % Document parameters
    \title{conor\_healy\_project\_1\_reflections\_document}
    
    
    

    % Pygments definitions
    
\makeatletter
\def\PY@reset{\let\PY@it=\relax \let\PY@bf=\relax%
    \let\PY@ul=\relax \let\PY@tc=\relax%
    \let\PY@bc=\relax \let\PY@ff=\relax}
\def\PY@tok#1{\csname PY@tok@#1\endcsname}
\def\PY@toks#1+{\ifx\relax#1\empty\else%
    \PY@tok{#1}\expandafter\PY@toks\fi}
\def\PY@do#1{\PY@bc{\PY@tc{\PY@ul{%
    \PY@it{\PY@bf{\PY@ff{#1}}}}}}}
\def\PY#1#2{\PY@reset\PY@toks#1+\relax+\PY@do{#2}}

\expandafter\def\csname PY@tok@w\endcsname{\def\PY@tc##1{\textcolor[rgb]{0.73,0.73,0.73}{##1}}}
\expandafter\def\csname PY@tok@c\endcsname{\let\PY@it=\textit\def\PY@tc##1{\textcolor[rgb]{0.25,0.50,0.50}{##1}}}
\expandafter\def\csname PY@tok@cp\endcsname{\def\PY@tc##1{\textcolor[rgb]{0.74,0.48,0.00}{##1}}}
\expandafter\def\csname PY@tok@k\endcsname{\let\PY@bf=\textbf\def\PY@tc##1{\textcolor[rgb]{0.00,0.50,0.00}{##1}}}
\expandafter\def\csname PY@tok@kp\endcsname{\def\PY@tc##1{\textcolor[rgb]{0.00,0.50,0.00}{##1}}}
\expandafter\def\csname PY@tok@kt\endcsname{\def\PY@tc##1{\textcolor[rgb]{0.69,0.00,0.25}{##1}}}
\expandafter\def\csname PY@tok@o\endcsname{\def\PY@tc##1{\textcolor[rgb]{0.40,0.40,0.40}{##1}}}
\expandafter\def\csname PY@tok@ow\endcsname{\let\PY@bf=\textbf\def\PY@tc##1{\textcolor[rgb]{0.67,0.13,1.00}{##1}}}
\expandafter\def\csname PY@tok@nb\endcsname{\def\PY@tc##1{\textcolor[rgb]{0.00,0.50,0.00}{##1}}}
\expandafter\def\csname PY@tok@nf\endcsname{\def\PY@tc##1{\textcolor[rgb]{0.00,0.00,1.00}{##1}}}
\expandafter\def\csname PY@tok@nc\endcsname{\let\PY@bf=\textbf\def\PY@tc##1{\textcolor[rgb]{0.00,0.00,1.00}{##1}}}
\expandafter\def\csname PY@tok@nn\endcsname{\let\PY@bf=\textbf\def\PY@tc##1{\textcolor[rgb]{0.00,0.00,1.00}{##1}}}
\expandafter\def\csname PY@tok@ne\endcsname{\let\PY@bf=\textbf\def\PY@tc##1{\textcolor[rgb]{0.82,0.25,0.23}{##1}}}
\expandafter\def\csname PY@tok@nv\endcsname{\def\PY@tc##1{\textcolor[rgb]{0.10,0.09,0.49}{##1}}}
\expandafter\def\csname PY@tok@no\endcsname{\def\PY@tc##1{\textcolor[rgb]{0.53,0.00,0.00}{##1}}}
\expandafter\def\csname PY@tok@nl\endcsname{\def\PY@tc##1{\textcolor[rgb]{0.63,0.63,0.00}{##1}}}
\expandafter\def\csname PY@tok@ni\endcsname{\let\PY@bf=\textbf\def\PY@tc##1{\textcolor[rgb]{0.60,0.60,0.60}{##1}}}
\expandafter\def\csname PY@tok@na\endcsname{\def\PY@tc##1{\textcolor[rgb]{0.49,0.56,0.16}{##1}}}
\expandafter\def\csname PY@tok@nt\endcsname{\let\PY@bf=\textbf\def\PY@tc##1{\textcolor[rgb]{0.00,0.50,0.00}{##1}}}
\expandafter\def\csname PY@tok@nd\endcsname{\def\PY@tc##1{\textcolor[rgb]{0.67,0.13,1.00}{##1}}}
\expandafter\def\csname PY@tok@s\endcsname{\def\PY@tc##1{\textcolor[rgb]{0.73,0.13,0.13}{##1}}}
\expandafter\def\csname PY@tok@sd\endcsname{\let\PY@it=\textit\def\PY@tc##1{\textcolor[rgb]{0.73,0.13,0.13}{##1}}}
\expandafter\def\csname PY@tok@si\endcsname{\let\PY@bf=\textbf\def\PY@tc##1{\textcolor[rgb]{0.73,0.40,0.53}{##1}}}
\expandafter\def\csname PY@tok@se\endcsname{\let\PY@bf=\textbf\def\PY@tc##1{\textcolor[rgb]{0.73,0.40,0.13}{##1}}}
\expandafter\def\csname PY@tok@sr\endcsname{\def\PY@tc##1{\textcolor[rgb]{0.73,0.40,0.53}{##1}}}
\expandafter\def\csname PY@tok@ss\endcsname{\def\PY@tc##1{\textcolor[rgb]{0.10,0.09,0.49}{##1}}}
\expandafter\def\csname PY@tok@sx\endcsname{\def\PY@tc##1{\textcolor[rgb]{0.00,0.50,0.00}{##1}}}
\expandafter\def\csname PY@tok@m\endcsname{\def\PY@tc##1{\textcolor[rgb]{0.40,0.40,0.40}{##1}}}
\expandafter\def\csname PY@tok@gh\endcsname{\let\PY@bf=\textbf\def\PY@tc##1{\textcolor[rgb]{0.00,0.00,0.50}{##1}}}
\expandafter\def\csname PY@tok@gu\endcsname{\let\PY@bf=\textbf\def\PY@tc##1{\textcolor[rgb]{0.50,0.00,0.50}{##1}}}
\expandafter\def\csname PY@tok@gd\endcsname{\def\PY@tc##1{\textcolor[rgb]{0.63,0.00,0.00}{##1}}}
\expandafter\def\csname PY@tok@gi\endcsname{\def\PY@tc##1{\textcolor[rgb]{0.00,0.63,0.00}{##1}}}
\expandafter\def\csname PY@tok@gr\endcsname{\def\PY@tc##1{\textcolor[rgb]{1.00,0.00,0.00}{##1}}}
\expandafter\def\csname PY@tok@ge\endcsname{\let\PY@it=\textit}
\expandafter\def\csname PY@tok@gs\endcsname{\let\PY@bf=\textbf}
\expandafter\def\csname PY@tok@gp\endcsname{\let\PY@bf=\textbf\def\PY@tc##1{\textcolor[rgb]{0.00,0.00,0.50}{##1}}}
\expandafter\def\csname PY@tok@go\endcsname{\def\PY@tc##1{\textcolor[rgb]{0.53,0.53,0.53}{##1}}}
\expandafter\def\csname PY@tok@gt\endcsname{\def\PY@tc##1{\textcolor[rgb]{0.00,0.27,0.87}{##1}}}
\expandafter\def\csname PY@tok@err\endcsname{\def\PY@bc##1{\setlength{\fboxsep}{0pt}\fcolorbox[rgb]{1.00,0.00,0.00}{1,1,1}{\strut ##1}}}
\expandafter\def\csname PY@tok@kc\endcsname{\let\PY@bf=\textbf\def\PY@tc##1{\textcolor[rgb]{0.00,0.50,0.00}{##1}}}
\expandafter\def\csname PY@tok@kd\endcsname{\let\PY@bf=\textbf\def\PY@tc##1{\textcolor[rgb]{0.00,0.50,0.00}{##1}}}
\expandafter\def\csname PY@tok@kn\endcsname{\let\PY@bf=\textbf\def\PY@tc##1{\textcolor[rgb]{0.00,0.50,0.00}{##1}}}
\expandafter\def\csname PY@tok@kr\endcsname{\let\PY@bf=\textbf\def\PY@tc##1{\textcolor[rgb]{0.00,0.50,0.00}{##1}}}
\expandafter\def\csname PY@tok@bp\endcsname{\def\PY@tc##1{\textcolor[rgb]{0.00,0.50,0.00}{##1}}}
\expandafter\def\csname PY@tok@fm\endcsname{\def\PY@tc##1{\textcolor[rgb]{0.00,0.00,1.00}{##1}}}
\expandafter\def\csname PY@tok@vc\endcsname{\def\PY@tc##1{\textcolor[rgb]{0.10,0.09,0.49}{##1}}}
\expandafter\def\csname PY@tok@vg\endcsname{\def\PY@tc##1{\textcolor[rgb]{0.10,0.09,0.49}{##1}}}
\expandafter\def\csname PY@tok@vi\endcsname{\def\PY@tc##1{\textcolor[rgb]{0.10,0.09,0.49}{##1}}}
\expandafter\def\csname PY@tok@vm\endcsname{\def\PY@tc##1{\textcolor[rgb]{0.10,0.09,0.49}{##1}}}
\expandafter\def\csname PY@tok@sa\endcsname{\def\PY@tc##1{\textcolor[rgb]{0.73,0.13,0.13}{##1}}}
\expandafter\def\csname PY@tok@sb\endcsname{\def\PY@tc##1{\textcolor[rgb]{0.73,0.13,0.13}{##1}}}
\expandafter\def\csname PY@tok@sc\endcsname{\def\PY@tc##1{\textcolor[rgb]{0.73,0.13,0.13}{##1}}}
\expandafter\def\csname PY@tok@dl\endcsname{\def\PY@tc##1{\textcolor[rgb]{0.73,0.13,0.13}{##1}}}
\expandafter\def\csname PY@tok@s2\endcsname{\def\PY@tc##1{\textcolor[rgb]{0.73,0.13,0.13}{##1}}}
\expandafter\def\csname PY@tok@sh\endcsname{\def\PY@tc##1{\textcolor[rgb]{0.73,0.13,0.13}{##1}}}
\expandafter\def\csname PY@tok@s1\endcsname{\def\PY@tc##1{\textcolor[rgb]{0.73,0.13,0.13}{##1}}}
\expandafter\def\csname PY@tok@mb\endcsname{\def\PY@tc##1{\textcolor[rgb]{0.40,0.40,0.40}{##1}}}
\expandafter\def\csname PY@tok@mf\endcsname{\def\PY@tc##1{\textcolor[rgb]{0.40,0.40,0.40}{##1}}}
\expandafter\def\csname PY@tok@mh\endcsname{\def\PY@tc##1{\textcolor[rgb]{0.40,0.40,0.40}{##1}}}
\expandafter\def\csname PY@tok@mi\endcsname{\def\PY@tc##1{\textcolor[rgb]{0.40,0.40,0.40}{##1}}}
\expandafter\def\csname PY@tok@il\endcsname{\def\PY@tc##1{\textcolor[rgb]{0.40,0.40,0.40}{##1}}}
\expandafter\def\csname PY@tok@mo\endcsname{\def\PY@tc##1{\textcolor[rgb]{0.40,0.40,0.40}{##1}}}
\expandafter\def\csname PY@tok@ch\endcsname{\let\PY@it=\textit\def\PY@tc##1{\textcolor[rgb]{0.25,0.50,0.50}{##1}}}
\expandafter\def\csname PY@tok@cm\endcsname{\let\PY@it=\textit\def\PY@tc##1{\textcolor[rgb]{0.25,0.50,0.50}{##1}}}
\expandafter\def\csname PY@tok@cpf\endcsname{\let\PY@it=\textit\def\PY@tc##1{\textcolor[rgb]{0.25,0.50,0.50}{##1}}}
\expandafter\def\csname PY@tok@c1\endcsname{\let\PY@it=\textit\def\PY@tc##1{\textcolor[rgb]{0.25,0.50,0.50}{##1}}}
\expandafter\def\csname PY@tok@cs\endcsname{\let\PY@it=\textit\def\PY@tc##1{\textcolor[rgb]{0.25,0.50,0.50}{##1}}}

\def\PYZbs{\char`\\}
\def\PYZus{\char`\_}
\def\PYZob{\char`\{}
\def\PYZcb{\char`\}}
\def\PYZca{\char`\^}
\def\PYZam{\char`\&}
\def\PYZlt{\char`\<}
\def\PYZgt{\char`\>}
\def\PYZsh{\char`\#}
\def\PYZpc{\char`\%}
\def\PYZdl{\char`\$}
\def\PYZhy{\char`\-}
\def\PYZsq{\char`\'}
\def\PYZdq{\char`\"}
\def\PYZti{\char`\~}
% for compatibility with earlier versions
\def\PYZat{@}
\def\PYZlb{[}
\def\PYZrb{]}
\makeatother


    % Exact colors from NB
    \definecolor{incolor}{rgb}{0.0, 0.0, 0.5}
    \definecolor{outcolor}{rgb}{0.545, 0.0, 0.0}



    
    % Prevent overflowing lines due to hard-to-break entities
    \sloppy 
    % Setup hyperref package
    \hypersetup{
      breaklinks=true,  % so long urls are correctly broken across lines
      colorlinks=true,
      urlcolor=urlcolor,
      linkcolor=linkcolor,
      citecolor=citecolor,
      }
    % Slightly bigger margins than the latex defaults
    
    \geometry{verbose,tmargin=1in,bmargin=1in,lmargin=1in,rmargin=1in}
    
    

    \begin{document}
    
    
    \maketitle
    
    

    
    \section{My Project: book\_manager.py}\label{my-project-book_manager.py}

\subsection{Summary}\label{summary}

\begin{itemize}
\tightlist
\item
  book\_manager.py is a tool to manage books, ratings, reviews and
  thoughts.
\item
  It includes 4 classes:
\item
  Bookshelf(): An entity that represents a place for all my books.
\item
  Book(): An entity that represents a book and associated data.
\item
  Review(): An entity that represents a book review and associated data.
\item
  BookshelfManager: An entity that creates new/loads instances of the
  bookshelf, book, and review objects and allows for user interaction.
\item
  I tried to manage data quality by managing user input at data entry,
  usually by requiring the user to input valid data before moving on.
  For this project, it was not practical to include validation for most
  text input fields (book title, Big Idea, etc.).
\item
  When book\_manager.py is loaded, all 4 classes are loaded and a single
  call is made to BookshelfManager(). From there, the user interacts
  with the BookshelfManager() instance.
\end{itemize}

\begin{enumerate}
\def\labelenumi{\arabic{enumi}.}
\tightlist
\item
  First, the user is provided with a description of the program and
  instructions to load or add a bookshelf.

  \begin{itemize}
  \tightlist
  \item
    The user can choose an existing bookshelf file to load. The
    bookshelf files are created using the pickle module and
    automatically saved by book\_manager.py with a ".bkshlf" extension.
  \item
    The user can also choose to add a new bookshelf and is prompted to
    name their bookshelf
  \end{itemize}
\item
  Once the bookshelf is loaded/created, the user is provided with the
  Main Menu and instructed to choose an option. The menu will repet
  until the user chooses a valid option. The user can Enter:

  \begin{itemize}
  \item
    \begin{enumerate}
    \def\labelenumii{(\Alph{enumii})}
    \setcounter{enumii}{11}
    \tightlist
    \item
      to (L)ist my books
    \end{enumerate}
  \item
    \begin{enumerate}
    \def\labelenumii{(\Alph{enumii})}
    \setcounter{enumii}{5}
    \tightlist
    \item
      to (F)ilter my books
    \end{enumerate}
  \item
    \begin{enumerate}
    \def\labelenumii{(\Alph{enumii})}
    \setcounter{enumii}{19}
    \tightlist
    \item
      to sor(T) my books
    \end{enumerate}
  \item
    \begin{enumerate}
    \def\labelenumii{(\Alph{enumii})}
    \setcounter{enumii}{6}
    \tightlist
    \item
      to (G)et the info about a book
    \end{enumerate}
  \item
    \begin{enumerate}
    \def\labelenumii{(\Alph{enumii})}
    \tightlist
    \item
      to (A)dd a book
    \end{enumerate}
  \item
    \begin{enumerate}
    \def\labelenumii{(\Alph{enumii})}
    \setcounter{enumii}{17}
    \tightlist
    \item
      to (R)eview a book
    \end{enumerate}
  \item
    \begin{enumerate}
    \def\labelenumii{(\Alph{enumii})}
    \setcounter{enumii}{2}
    \tightlist
    \item
      to save and (C)lose
    \end{enumerate}
  \item
    \begin{enumerate}
    \def\labelenumii{(\Alph{enumii})}
    \setcounter{enumii}{16}
    \tightlist
    \item
      to (Q)uit without saving
    \end{enumerate}
  \item
    \begin{enumerate}
    \def\labelenumii{(\Alph{enumii})}
    \setcounter{enumii}{18}
    \tightlist
    \item
      to (S)ettings
    \end{enumerate}
  \end{itemize}
\item
  Each menu option has unique behavior:

  \begin{itemize}
  \item
    \begin{enumerate}
    \def\labelenumii{(\Alph{enumii})}
    \setcounter{enumii}{11}
    \tightlist
    \item
      to (L)ist my books
    \end{enumerate}

    \begin{itemize}
    \tightlist
    \item
      Prints out the books in an ascii bookshelf
    \end{itemize}
  \item
    \begin{enumerate}
    \def\labelenumii{(\Alph{enumii})}
    \setcounter{enumii}{5}
    \tightlist
    \item
      to (F)ilter my books
    \end{enumerate}

    \begin{itemize}
    \tightlist
    \item
      Asks the user to choose a method to filter the books
    \item
      Multiple filters can be applied. Each selection in the filter menu
      is applied against the books remaining after previous filters.
    \item
      Also, users can reset all filters and sorts or back out to the
      Main Menu
    \end{itemize}
  \item
    \begin{enumerate}
    \def\labelenumii{(\Alph{enumii})}
    \setcounter{enumii}{19}
    \tightlist
    \item
      to sor(T) my books
    \end{enumerate}

    \begin{itemize}
    \tightlist
    \item
      Asks the user to choose a method to sort the books. Once a sort is
      chosen, the user is prompted to choose whether to sort ascending
      or descending
    \item
      Multiple sorts can be applied. Each selection in the filter menu
      is applied against the book list sorted by previous sorts. If one
      wanted to sort by multiple sorts intentionally (e.g. by Author
      then by Title), then the sorts should be applied backwards (First
      by Title, then by Author) to get the right result.
    \item
      Also, users can back out to the Main Menu.
    \end{itemize}
  \item
    \begin{enumerate}
    \def\labelenumii{(\Alph{enumii})}
    \setcounter{enumii}{6}
    \tightlist
    \item
      to (G)et the info about a book
    \end{enumerate}

    \begin{itemize}
    \tightlist
    \item
      Prints out the books in an ascii bookshelf
    \item
      Asks the user to choose a book based on it's position in the
      bookshelf
    \item
      Prints details about the book
    \end{itemize}
  \item
    \begin{enumerate}
    \def\labelenumii{(\Alph{enumii})}
    \tightlist
    \item
      to (A)dd a book
    \end{enumerate}

    \begin{itemize}
    \tightlist
    \item
      Prompts the user to add a book.
    \item
      If "Want to Review" is chosen as the status, the user is prompted
      for whether they want to review the book now.

      \begin{itemize}
      \tightlist
      \item
        If yes, the user is prompted to add a review
      \item
        If no, the book status is changed to "Read, but not reviewed"
      \end{itemize}
    \end{itemize}
  \item
    \begin{enumerate}
    \def\labelenumii{(\Alph{enumii})}
    \setcounter{enumii}{17}
    \tightlist
    \item
      to (R)eview a book
    \end{enumerate}

    \begin{itemize}
    \tightlist
    \item
      Prints out the books in an ascii bookshelf
    \item
      Asks the user to choose a book based on it's position in the
      bookshelf
    \item
      Prompts the user to add a review.
    \end{itemize}
  \item
    \begin{enumerate}
    \def\labelenumii{(\Alph{enumii})}
    \setcounter{enumii}{2}
    \tightlist
    \item
      to save and (C)lose
    \end{enumerate}

    \begin{itemize}
    \tightlist
    \item
      saves the bookshelf with the pickle method using the name of the
      bookshelf and a ".bkshlf" file extension
    \item
      Exits the program
    \end{itemize}
  \item
    \begin{enumerate}
    \def\labelenumii{(\Alph{enumii})}
    \setcounter{enumii}{16}
    \tightlist
    \item
      to (Q)uit without saving
    \end{enumerate}

    \begin{itemize}
    \tightlist
    \item
      Prompts the user to enter "Quit"

      \begin{itemize}
      \tightlist
      \item
        If the user enters "Quit", exits the program.
      \item
        If not, returns to the Main Menu
      \end{itemize}
    \end{itemize}
  \item
    \begin{enumerate}
    \def\labelenumii{(\Alph{enumii})}
    \setcounter{enumii}{18}
    \tightlist
    \item
      to (S)ettings
    \end{enumerate}

    \begin{itemize}
    \tightlist
    \item
      Prompts the user with choices to to:

      \begin{itemize}
      \tightlist
      \item
        Reset the bookshelf by removing all filters and sort and
        recalculating all counters.
      \item
        See any Beta test features (currently a printout of the
        bookshelf, books, an example book, and the bookshelf associated
        wiht the example book)
      \item
        Back out to the Main Menu.
      \end{itemize}
    \end{itemize}
  \end{itemize}
\end{enumerate}

\subsection{Testing \& Using
book\_manager.py:}\label{testing-using-book_manager.py}

\begin{itemize}
\tightlist
\item
  Hopefully, the Summary above covers the functionality pretty well. A
  simple \texttt{python\ book\_manager.py} command should load and run
  the project.
\item
  I created this project using a Mac, but don't anticipate any problems
  running it on windows or linux.
\item
  I am providing a demo bookshelf (demo.bkshlf) pre-loaded with 25 books
  and 4 reviews. The file can be loaded by choosing (L) at the intial
  prompt, then typing the path to the file.
\item
  A new bookshelf can also be created. Be careful to name it something
  other than "demo" or the file I provided will be overwritten.
\end{itemize}

\subsection{Challenges}\label{challenges}

\begin{itemize}
\tightlist
\item
  This was a great project that helped me learn quite a bit about python
  and object oriented programming.
\item
  I also tried my hand at git branching, and think it could be
  recommended for students in the future. Class development seems ideal
  as a toy example for feature development, and a reference document /
  optional async would have been useful.
\item
  My biggest challenge was project scope. Adding book details with user
  prompts + sort + filters + printing out bookshelves, book details, and
  review details made for a bigger project that first anticipated. (I
  got instructor approval to go over the 750 line limit to get it all
  in, and still left some valuable features on the Future Enhancements
  list).
\item
  My second biggest challenge was workflow related. We haven't covered
  use of an IDE in this course, but jupyter notebooks are not well
  suited for this kind of development. I ended up with a hybrid workflow
  that got me done, but I look forward to finding a better tool for
  development work.
\item
  I tried briefly to get Goodreads integration working, but I found that
  the Goodreads import and export .csv files are not compatible and are
  not documented. I plan to play with the import functionality in the
  future, and then build a Save to Goodreads.csv option once I
  understand what Goodreads wants as an import.
\item
  I ran into bugs while developing the project, but most of my issues
  seemed to be about the quirks of this project that would likely be
  replaced by more sophisticated tools in production:
\item
  building the ascii bookshelf to display the books
\item
  setting up the multi\_input function to allow users to enter multiple
  lines of input.
\item
  There are a few places where I use a long series of elif statements to
  dynamically return different attributes/results, and I'm sure there's
  a better way to code them. However, it is functional so I chose to
  focus elsewhere.
\end{itemize}


    % Add a bibliography block to the postdoc
    
    
    
    \end{document}
